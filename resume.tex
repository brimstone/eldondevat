\documentclass[letterpaper,11pt]{article}

\usepackage{latexsym}
\usepackage[empty]{fullpage}
\usepackage{titlesec}
\usepackage{marvosym}
\usepackage[usenames,dvipsnames]{color}
\usepackage{verbatim}
\usepackage{enumitem}
\usepackage[hidelinks]{hyperref}
\usepackage{fancyhdr}
\usepackage[english]{babel}
\usepackage{tabularx}
\usepackage{hyphenat}

\addtolength{\topmargin}{-.5in}
\pdfgentounicode=1
\pagestyle{fancy}
\fancyhf{} % clear all header and footer fields
\fancyfoot{}
\renewcommand{\headrulewidth}{0pt}
\renewcommand{\footrulewidth}{0pt}
\titleformat{\section}{
  \vspace{-4pt}\scshape\raggedright\large
}{}{0em}{}[\color{black}\titlerule\vspace{-5pt}]



\begin{document}
\begin{center}

  \textbf{\LARGE \scshape Eldon Stegall } \\
	\large
	Functionality, Security, Reliability \\
	\vspace{+3pt}
	\small
  \hspace{.5pt} \href{https://github.com/eldondevat}{\color{blue}GitHub:eldondevat}
  \hspace{.5pt} \href{mailto:esrtxr@eldondevcom}{\color{blue}esrtxr@eldondevcom}
  \hspace{.5pt} \href{tel:+1-404-492-6299}{\color{blue}+1-404-492-6299}

\end{center}
\section{Objective}
To contribute to software systems that have a positive impact on the users and the world around them, via
\begin{itemize}
	\item Communicating with internal and external stakeholders to understand requirements and evaulate strategic and technical approaches to success.
  \item Designing, implementing, and operationalizing software that fulfills a user need safely and reliably.
  \item Applying best practices related to the engineering and security of code, databases, virtual and physical infrastructure, and organizational effectiveness.
\end{itemize}
\section{Recent Experience} 
Having worked at startups, I have maintained a
career-long passion for shipping code early, and developed the experience to
utilize powerful modern tooling to do so safely and effectively. I have
spearheaded such recent projects as:
\begin{itemize} 
\item A comprehensive embedded firmware system built with the Yocto toolkit. Builds were done in the cloud via Github automation, and could be deployed to test devices minutes after the code changes were pushed. The target architecture
changed in the middle of the project (aarch64 to armv7l), and we were able to quickly and effectively retarget to the new architecture. \textcolor{BlueViolet}{Effective Enginnering}
\item An IOT agent communicating securely via mTLS with AWS IOT, and integrating with a variety of cloud functionality for video processing, observability, and cloud control. \textcolor{BrickRed}{Security}, \textcolor{BlueViolet}{Effective Enginnering}
\item A break-glass disaster recovery system and strategy to enable IOT device recovery in the event of upstream provider outages. \textcolor{ForestGreen}{Reliability}
\item A terraform strategy for deploying per-endpoint AWS lambda functions, decreasing the time to deploy new API endpoints to less than a day, including per-endpoint monitoring and Authn/Authz. \textcolor{BrickRed}{Security} \textcolor{ForestGreen}{Reliability}
\item A Kubernetes operator enabling TLS certificate provisioning to secure communications into and between nodes of a distributed database. \textcolor{BrickRed}{Security}
\item A bluetooth-enabled Flutter application. Linting, testing, and internal releases were performed continuously for both android and IOS via Github automation. \textcolor{BrickRed}{Security}, \textcolor{BlueViolet}{Effective Enginnering}
\end{itemize}
\
\end{document}

