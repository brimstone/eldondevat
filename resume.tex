\documentclass[letterpaper,10pt]{article}

\usepackage{latexsym}
\usepackage[empty]{fullpage}
\usepackage{titlesec}
\usepackage{marvosym}
\usepackage[usenames,dvipsnames]{color}
\usepackage{verbatim}
\usepackage{enumitem}
\usepackage[hidelinks]{hyperref}
\usepackage{fancyhdr}
\usepackage[english]{babel}
\usepackage{tabularx}
\usepackage{hyphenat}
\usepackage{geometry}
\geometry{
textwidth=550pt,
textheight=700pt,
}
\usepackage{layout}


\pdfgentounicode=1
\pagestyle{fancy}
\fancyhf{} % clear all header and footer fields
\fancyfoot{}
\setlength{\footskip}{3.60004pt}
\renewcommand{\headrulewidth}{0pt}
\renewcommand{\footrulewidth}{0pt}
\titleformat{\section}{
  \vspace{-6pt}\scshape\raggedright\large
}{}{0em}{}[\color{black}\titlerule\vspace{-5pt}]
\setlist[itemize]{leftmargin=*}
\begin{document}

\begin{center}
	\textbf{\LARGE Eldon Stegall } \\
	\large
	Security, Reliability, Engineering Effectiveness \\
	\vspace{+3pt}
	\small
	\hspace{.5pt} \href{https://github.com/eldondevat}{\color{blue}GitHub:eldondevat}
	\hspace{.5pt} \href{mailto:esrtxr@eldondevcom}{\color{blue}esrtxr@eldondevcom}
	\hspace{.5pt} \href{tel:+1-404-492-6299}{\color{blue}+1-404-492-6299}
\end{center}


\section{Objective}
\small
\newsavebox{\objbox}
\sbox{\objbox}{
	\begin{minipage}[b]{0.6\textwidth}
		To contribute to software systems that have a positive impact on the users and the world around them, via
		\begin{itemize}
			\setlength\itemsep{0.1em}
			\item Communicating with internal and external stakeholders to understand requirements and evaluate strategic and tactical approaches to technological success.
			\item Designing, implementing, and operationalizing software that fulfills a user need safely and reliably.
			\item Applying best practices related to the engineering and security of code, databases, virtual and physical infrastructure, and organizational effectiveness.
		\end{itemize}
	\end{minipage}
}
\usebox{\objbox}\hfill
\fbox{
	\begin{minipage}[b]{0.36\textwidth}
		\small\raggedright
		Recent positions held, references and \\ comprehensive 18 year professional software history available on request.
		\begin{itemize}
			\item Keep Technologies, 2021 - 2023, Engineer
			\item Roadie, 2021, Senior Software Engineer
			\item DataStax, 2018 - 2020, \mbox{Security Architect}
		\end{itemize}
	\end{minipage}
}

\section{Recent Experience}
Enjoying the agility of startups, I have maintained a
career-long passion for shipping functional systems early, and developed the experience to
utilize powerful modern tooling to do so safely and effectively. I have
worked with teams to spearheaded projects such as
\begin{itemize}
	\setlength\itemsep{0.1em} \item A comprehensive embedded firmware system
	built with the Yocto toolkit. Builds were done in AWS via custom Github
	automation, and could be deployed to test devices minutes after the code
	changes were pushed. The target architecture
	      changed in the middle of the project (aarch64 to armv7l), and we were able to quickly and effectively retarget to the new architecture due to engineering foresight. \textbf{Effective Engineering} \textit{Keep Technologies, 2023}
	\item An IOT agent communicating securely via mTLS with AWS IOT, and integrating with a variety of cloud functionality for video processing, observability, and cloud control. \textbf{Security}, \textbf{Effective Engineering} \textit{Keep Technologies, 2022}
	\item A break-glass disaster recovery system and strategy to enable IOT device recovery in the event of upstream provider outages. \textbf{Reliability} \textit{Keep Technologies, 2021}
	\item A Terraform strategy for building and deploying per-endpoint AWS lambda functions, decreasing the time to deploy new API endpoints to less than a day, including per-endpoint monitoring and Authn/Authz. \textbf{Security}, \textbf{Reliability} \textit{Keep Technologies, 2021}
	\item Multiteam migration of microservices and databases from Heroku to EKS, including containerization and multi-environment operation. \textbf{Reliability} \textit{Roadie, 2021}
	\item A Kubernetes operator to deploy a distributed database, enabling TLS certificate provisioning to secure communications into and between nodes. \textbf{Security} \textit{DataStax, 2020}
	\item Vulnerability management, incident response coordination, SOC2 controls design, and pentest management. \textbf{Security} \textit{DataStax, 2020}
\end{itemize}


\section{Technology Focus}
In order to deliver software to users, I typically use technology and tooling such as
\begin{itemize}
	\setlength\itemsep{0.1em}
	\item Github Issues, JIRA, or Shortcut to manage requirements and feature development, which I often model in
	\item OpenAPI or GRPC in order to leverage the extensive software ecosystems around each to scaffold software faster, which I usually implement in
	\item Golang or Flutter to build business logic and implement tests, services, and clients. I have extensive experience in Python, Ruby, JavaScript, and Java, and will use them when they are the right choice or already filling the need.
	\item Github Actions, Gitlab Pipelines, or Jenkins usually lint and build the code I write, execute tests, and generate container images, and scan for security posture adherence to deploy into
	\item Kubernetes or Terraform to deploy software to a cloud provider in execution environments such as Lambda, EKS or GKE. I have also worked with operating Kubernetes as a platform on bare metal in production environments.
\end{itemize}

\section{Additional Topics, Accomplishments, And Efforts}
Adjacent to my daily work, I improve my craft and contribute back to the community in a variety of ways, including
\begin{itemize}
	\setlength\itemsep{0.1em}
	\item BSc Computer Science, BSc Discrete Mathematics, Ga Tech, 2006
	\item Presenting at BSides Atlanta 2020 as a remote speaker on \href{https://www.youtube.com/watch?v=kLCaAaUd1mM}{\color{blue}Cloud Security tooling}
	\item Contributing improvements and fixes to projects such as the AWS Terraform provider, Wireshark, and Qemu.
	\item Obtaining the Certified Kubernetes Administrator recognition from the Cloud Native Computing Foundation.
	\item Deploying and maintaining the installation of Open Source video conferencing technology enabling remote meetings of the Atlanta Cybersecurity community facilitating presentations by numerous Cybersecurity professionals to the public.
\end{itemize}
%\layout*
\end{document}

